\section{Conclusion}

In this paper, we showed how to formulate a novel queue transmission model (QTM)
model of traffic flow with non-homogeneous time steps as a linear program.  We
then proceeded to allow the traffic signals to become discrete variables subject
to a delay minimizing optimization objective and standard traffic signal
constraints leading to a final MILP formulation of traffic signal control.  We
experimented with this novel QTM-based MILP control in a range of networks and
demonstrated that by exploiting the non-homogeneous time steps supported by the
QTM, we are able to scale the model up to larger networks whilst maintaining the
same quality of a homogeneous solution using more binary
variables. Altogether, this work represents a
major step forward in the scalability of MILP-based jointly optimized traffic
signal control via the use of a non-homogeneous traffic models and thus helps
pave the way for fully optimized joint urban traffic signal controllers as an
improved successor technology to existing signal control methods.

%We have demonstrated that by exploiting the non-homogeneous time steps
%supported by the QTM, we are able to scale the model up to larger
%networks and using the same number of binary variables as a
%homogeneous time step, and with the same quality of a homogeneous
%solution using more binary variables.
