\section*{Abstract}

% Basically just the topic sentences of the Intro. :)  < 250 words

We build on the body of work in mixed integer linear programming
(MILP) approaches that attempt to jointly optimize traffic signal
control over an \emph{entire traffic network} (rather than focus on
arterial routes) and specifically on improving the scalability of
these methods for large urban traffic networks.  Our primary insight
in this work stems from the fact that MILP-based approaches to traffic
control used in a receding horizon control manner (that replan at
fixed time intervals) need only plan with high fidelity control
policies in the early stages of the signal plan and can use a coarser
time step to ``see'' over a long horizon to preemptively adapt to
distant platoons and other predicted long-term changes in traffic
flows.  To this end, we contribute the queue transmission model (QTM)
which blends elements of cell-based and link-based modeling approaches
to to enable a non-homogenous MILP formulation of traffic signal
control.  We then experiment with this novel QTM-based MILP control in
a range of networks demonstrating the improved scalability possible
with non-homogeneous time steps in comparison to the best homogenous
time step and also the scalability of the model --- we demonstrate
near-optimal traffic control models over longer horizons and larger
networks than shown in previous implementations of MILP-based traffic
signal control.

\newcommand{\wordsInAbstract}{
    \immediate\write18{texcount -sum -1 tex/Abstract.tex > 'countInAbstract.txt'}
    \IfFileExists{./countInAbstract.txt}{\input{countInAbstract.txt}}{??}}
\fnremark{Using \wordsInAbstract{} words up to here. Maximum is 250 words.

Make sure to follow instructions and author guide:
  \url{http://onlinepubs.trb.org/onlinepubs/AM/InfoForAuthors.pdf}
  \url{http://onlinepubs.trb.org/onlinepubs/am/2015/WritingForTheTRRecord.pdf}

Also note this example related paper from Steve Smith (formatted to TRB specs):
  \url{https://www.ri.cmu.edu/pub_files/2014/1/TRB14UTC.pdf}
}

