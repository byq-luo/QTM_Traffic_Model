\section*{Abstract}

% Basically just the topic sentences of the Intro. :)  < 250 words

% Motivation, quantify

As urban traffic congestion is on the increase worldwide, it is
critical to maximize capacity and throughput of existing road
infrastructure through optimized traffic signal control.  To this end,
we build on the body of work in mixed integer linear programming
(MILP) approaches that attempt to jointly optimize traffic signal
control over an \emph{entire traffic network} and specifically on
improving the scalability of these methods for large numbers of
intersections.  Our primary insight in this work stems from the fact
that MILP-based approaches to traffic control used in a receding
horizon control manner (that replan at fixed time intervals) need to
compute high fidelity control policies only for the early stages of
the signal plan; therefore, coarser time steps can be employed to
``see'' over a long horizon to preemptively adapt to distant platoons
and other predicted long-term changes in traffic flows.  To this end,
we contribute the queue transmission model (QTM) which blends elements
of cell-based and link-based modeling approaches to enable a
non-homogeneous time MILP formulation of traffic signal control.
%
We then experiment with this novel QTM-based MILP control in a range
of networks demonstrating substantially improved scalability and
traffic signal control quality when using non-homogeneous time steps
in comparison to homogeneous time steps.
%
%% I think the following statement I wrote is too risky to say given that
%% - the traffic withholding problem may occur
%% - we are using the major/minor frame approach
%% - we are somehow claiming that the QTM is an accurate model of real traffic flow
%% - Gurobi may have a time-limit for some results
%Our experiments also provide near-optimal traffic control policies
%for dense traffic networks up to nine intersections than shown in previous
%implementations of MILP-based traffic signal control.


%\newcommand{\wordsInAbstract}{
    \immediate\write18{texcount -sum -1 tex/Abstract.tex > 'countInAbstract.txt'}
    \IfFileExists{./countInAbstract.txt}{\input{countInAbstract.txt}}{??}}
\fnremark{Using \wordsInAbstract{} words up to here. Maximum is 250 words.

Make sure to follow instructions and author guide:
  \url{http://onlinepubs.trb.org/onlinepubs/AM/InfoForAuthors.pdf}
  \url{http://onlinepubs.trb.org/onlinepubs/am/2015/WritingForTheTRRecord.pdf}

Also note this example related paper from Steve Smith (formatted to TRB specs):
  \url{https://www.ri.cmu.edu/pub_files/2014/1/TRB14UTC.pdf}
}

