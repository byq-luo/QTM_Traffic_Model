\section{Introduction}

{\bf Previous content:}

This report describes a new model for optimised traffic signal
planning using a MILP formulation. Previously,
\trbcite{lin2004enhanced} describe a MILP formulation for optimised
traffic signal planning based on the Cell Transmission Model of
\trbcite{daganzo1995cell}. However a CTM based model is limited in
scalability by the requirement that each road way in the network must
be partitioned up into segments that take exactly $\DT[]=1$ to
traverse at the free flow speed. We get around this limitation by
using a queue based model that supports non-homogeneous time steps,
and we will show how we can exploit this property to scale a network
without significant increase in the number of variables or loss of
quality.

{\bf Scott's suggested argument structure:}

\remark{I say we drop the issue of many segments and simply criticize
  the CTM by saying that it is not easy to make non-homogenous and
  that is what we need?  In general, the Intro probably has to push on
  the non-homogenous motivation a little earlier and forcefully since
  we're claiming *this* is the reason for the paper, the QTM and
  ultimately our evaluation comparing homogenous vs. non-homogenous.
  We may have broader (not fully validated) motivations but we need a
  watertight argument and paper.}

In practice, control of congested urban traffic networks ranges from
fixed time to actuated to adaptive (SCATS, SCOOT).  However there is
further opportunity to improve traffic scheduling through the use of
optimization techniques as proposed in various works
(Gartner, Lin and Wang, MARLIN -- Toronto, SURTRAC -- Smith).  In this
work, we specifically build on the MILP approach to traffic
optimization that extends previous work by Lin and Wang who
optimize traffic signals (discrete choices at each time step) in a 
Cell Transmission Model (CTM)~\trbcite{daganzo1995cell} of traffic flow.
Unfortunately a CTM based model is limited in
scalability by the requirement that each road way in the network must
be partitioned up into segments that take exactly $\DT[]=1$ to
traverse at the free flow speed.
This leads to a large number of cells
(and hence variables in the MILP) and it makes it difficult to
allow for variable length (non-homogenous) time steps in the model
updates that we demonstrate in this work
to be useful for optimized traffic signal planning over long horizons.

To allow for more parsimonious models of traffic flow and non-homogenous
time steps for updates, we introduce the queue transmission model (QTM) [succinct description needed], which offers the following
benefits:
\begin{itemize}
\item non-homogenous time steps which can keep the model compact over long horizons
\item models platoons without fine-grained cell-based road model or explicit variables to model platoons -- underlies a key precept in urban traffic control to let platoons pass through without stopping,
\item still leads to a MILP, albeit one with much fewer variables owing to the reduced number of queues vs. cells and the reduced number of time steps.
\end{itemize}
Some drawbacks -- does not capture nonlinear flow-density relationship of the
fundamental diagram.  \remark{Argue why this is OK for saturated
  urban networks at peak times.  Green waves in SCATS/SCOOT operate on
  principle of a fixed propagation time, so does Gartner's MILP model.
Captures the right level of detail without blowing up size of model.}

In the remainder of this paper, we show first how to formulate traffic flow
with fixed signals as an LP in the QTM; we then proceed to allow the
traffic signals to become discrete variables subject to optimization leading
to our final MILP formulation of traffic signal control.  We then experiment
with QTM-based control in a range of networks demonstrating the improved
scalability possible with non-homogeneous time steps in comparison to
the best homogenous time step and also the scalability of the model ---
we demonstrate near-optimal traffic control models over longer horizons and
larger networks than shown in previous implementations of MILP-based
traffic signal control.

\remark{Paper should follow the basic progression outlined above.  We
  could really use some pictures in the Intro to refer to here and
  subsequently -- both a traffic network divided into queues, and the
  concept of the piecewise linear evolution of traffic flow with
  non-homogenous time steps that I had provided in my early writeup.
  I think these help visually explain much of the context for the
  paper and its approach.}

