\section{Introduction}

%{\bf Previous content:}
%
%This report describes a new model for optimised traffic signal
%planning using a MILP formulation. Previously,
%\trbcite{lin2004enhanced} describe a MILP formulation for optimised
%traffic signal planning based on the Cell Transmission Model of
%\trbcite{daganzo1995cell}. However a CTM based model is limited in
%scalability by the requirement that each road way in the network must
%be partitioned up into segments that take exactly $\DT[]=1$ to
%traverse at the free flow speed. We get around this limitation by
%using a queue based model that supports non-homogeneous time steps,
%and we will show how we can exploit this property to scale a network
%without significant increase in the number of variables or loss of
%quality.
%
%{\bf Scott's suggested argument structure:}
%
%\remark{I say we drop the issue of many segments and simply criticize
%  the CTM by saying that it is not easy to make non-homogenous and
%  that is what we need?  In general, the Intro probably has to push on
%  the non-homogenous motivation a little earlier and forcefully since
%  we're claiming *this* is the reason for the paper, the QTM and
%  ultimately our evaluation comparing homogenous vs. non-homogenous.
%  We may have broader (not fully validated) motivations but we need a
%  watertight argument and paper.}

As cities rapidly grow in population while urban traffic
infrastructure often adapts at a slower pace, it is critical to
maximize capacity and throughput of existing road infrastructure
through optimized traffic signal control.  Unfortunately, many large
cities still use some degree of \emph{fixed-time} control (e.g.,
Toronto~\trbcitenum{el2013multiagent}) even if they also use
\emph{actuated} or \emph{adaptive} control methods such as SCATS~\trbcitenum{scats80}
or SCOOT~\trbcitenum{scoot81}.  However there is
further opportunity to improve traffic signal control even beyond
adaptive methods through the use of \emph{optimized} controllers as
evidenced in a variety of approaches ranging from mixed integer
(linear)
programming~\trbcitenum{gartner1974optimization,gartner2002arterial,lo1998novel,he2011pamscod,lin2004enhanced,han2012link}
to heuristic search~\trbcitenum{lo1999dynamic,he2010heuristic} to
scheduling~\trbcitenum{smith2013surtrac} to reinforcement
learning~\trbcitenum{el2013multiagent}.  While such optimized controllers
hold the promise of maximizing existing infrastructure capacity by
finding more complex (and potentially more optimal) jointly
coordinated intersection policies than arterially-focused master-slave
approaches such as SCATS and SCOOT, such optimized methods are
computationally demanding
%compared to adpative control methods such as SCOOT and SCATS
%(which can run on 1970's error hardware)
and either (a) do not guarantee jointly optimal solutions over a large
intersection network (often because they only consider coordination of
neighboring intersections or arterial routes) or (b) fail to scale to
large intersection networks simply for computational reasons (which is
the case for many mixed integer programming approaches).

In this work, we build on the body of work in mixed integer linear
programming (MILP) approaches that attempt to jointly optimize traffic
signal control over an \emph{entire traffic network} (rather than
focus on arterial routes) and specifically on improving the
scalability of these methods for large urban traffic networks.  In our
investigation of existing approaches in this vein, namely exemplar
methods in the spirit of~\trbcitenum{lo1998novel,lin2004enhanced,han2012link} that
use a (modified) cell transmission model
(CTM)~\trbcitenum{daganzo1994cell,daganzo1995cell} for their underlying
prediction of traffic flows, we remark that a major drawback is the
CTM-imposed requirement to choose a predetermined homogenous (and
often necessarily small) time step for reasonable modeling fidelity.
The need to model large number of CTM cells with a small time step
however leads to MILPs that are exceedingly large and intractable to
solve for large traffic networks.

Our primary insight in this work stems from the fact that MILP-based
approaches to traffic control used in a receding horizon control
manner (that replan at fixed time intervals) need only plan with high
fidelity control policies in the early stages of the signal plan and
can use a coarser time step to ``see'' over a long horizon to
preemptively adapt to distant platoons and other predicted long-term
changes in traffic flows.  This need for non-homogenous control in
turn spawns the need for an additional innovation: we require a
traffic flow simulation model that permits non-homogenous time steps
and properly models the travel time delay between lights.  To this
end, we might consider CTM extensions such as the variable cell length
CTM~\trbcitenum{xiaojian2010urban}, stochastic CTM
extensions~\trbcitenum{sumalee2011stochastic,jabari2012stochastic},
extensions for better modeling freeway-urban
interactions~\trbcitenum{huang2011traffic} including CTM hybrids with
link-based models~\trbcitenum{muralidharan2009freeway}, assymmetric CTMs
for better handling flow imbalances in merging
roads~\trbcitenum{gomes2006optimal}, the situational CTM for better
modeling of boundary conditions~\trbcitenum{kim2002online}, and the
lagged CTM for improved modeling of the flow density
relation~\trbcitenum{lu2011discrete}.  However despite the widespread
varieties of the CTM and the usage of the
CTM~\trbcitenum{alecsandru2011assessment} for a range of applications,
there seems to be no extension that permits non-homogeneous time steps
as required in our novel MILP-based control approach.

For this reason, as a major contribution of this work to enable our
non-homogenous time MILP-based model of joint intersection control, we
contribute the queue transmission model (QTM) which blends elements of
cell-based and link-based modeling approaches with the following key
benefits: (a) unlike previous joint intersection control work~\trbcitenum{lo1998novel,lin2004enhanced,han2012link},
it is inherently intended for \emph{non-homogenous} time steps that can be used for control over large horizons,
(b) any length of roadway with no merges or diverges can be modeled as
a single queue leading to compact models of large traffic networks thus maintaining relatively
compact MILPS for large traffic networks (i.e., larges numbers of cells are not required between intersections),
(c) it accurately models fixed travel time delays critical to green
wave coordination as
in~\trbcitenum{gartner1974optimization,gartner2002arterial,he2011pamscod}
through the use of a non-first order Markovian update model and
combines this with the more global intersection signal optimization
approach of~\trbcitenum{lo1998novel,lin2004enhanced,han2012link}.

In the remainder of this paper, we show first how to formulate our
novel QTM model of traffic flow with non-homogenous time steps as a
linear program.  We then proceed to allow the traffic signals to
become discrete variables subject to a delay minimizing optimization
objective and standard cycle and phase time constraints leading to our
final MILP formulation of traffic signal control.  We then experiment
with this novel QTM-based MILP control in a range of networks
demonstrating the improved scalability possible with non-homogeneous
time steps in comparison to the best homogenous time step and also the
scalability of the model --- we demonstrate near-optimal traffic
control models over longer horizons and larger networks than shown in
previous implementations of MILP-based traffic signal control.

%%%%%%%%%%%%%%%%%%%%%%%%%%%%%%%%%%%%%%%%%%%%%%%%%%%%%%%%%%%%%%%%%%%%%%%%%%

\remark{Paper should follow the basic progression outlined in last paragraph above.
  Need to be careful to maintain the thread of the story throughout the paper and
  the summarize it in the conclusion with the major take-home results --- longer
  horizons and larger networks for MILP-based control!}

\remark{ We could really use some pictures in the Intro to refer to
  here and subsequently -- both a traffic network divided into queues,
  and the concept of the piecewise linear evolution of traffic flow
  with {\bf non-homogenous} (dilated) time steps, something like I had
  provided in my early writeup.  I think these help visually explain
  much of the context for the paper and its approach and are critical
  for reviewer understanding on a time budget for reading this They
  may only read the first 2-3 pages and then skim!}

\remark{A picture is worth
  a 1000 words but we only pay 250, hence a 4X ROI on pictures!}

%%%%%%%%%%%%%%%%%%%%%%%%%%%%%%%%%%%%%%%%%%%%%%%%%%%%%%%%%%%%%%%%%%%%%%%%%%
%%%%%%%%%%%%%%%%%%%%%%%%%%%%%%%%%%%%%%%%%%%%%%%%%%%%%%%%%%%%%%%%%%%%%%%%%%
%%%%%%%%%%%%%%%%%%%%%%%%%%%%%%%%%%%%%%%%%%%%%%%%%%%%%%%%%%%%%%%%%%%%%%%%%%

\comment{
optimization techniques as
proposed in various works~\cite (Gartner, Lin and Wang, MARLIN --
Toronto, SURTRAC -- Smith).  In this work, we specifically build on
the MILP approach to traffic optimization that extends previous work
by Lin and Wang who optimize traffic signals (discrete choices at each
time step) in a Cell Transmission Model
(CTM)~\trbcite{daganzo1995cell} of traffic flow.  Unfortunately a CTM
based model is limited in scalability by the requirement that each
road way in the network must be partitioned up into segments that take
exactly $\DT[]=1$ to traverse at the free flow speed.  This leads to a
large number of cells (and hence variables in the MILP) and it makes
it difficult to allow for variable length (non-homogenous) time steps
in the model updates that we demonstrate in this work to be useful for
optimized traffic signal planning over long horizons.

To allow for more parsimonious models of traffic flow and non-homogenous
time steps for updates, we introduce the queue transmission model (QTM) [succinct description needed], which offers the following
benefits:
\begin{itemize}
\item non-homogenous time steps which can keep the model compact over long horizons
\item models platoons without fine-grained cell-based road model or explicit variables to model platoons -- underlies a key precept in urban traffic control to let platoons pass through without stopping,
\item still leads to a MILP, albeit one with much fewer variables owing to the reduced number of queues vs. cells and the reduced number of time steps.
\end{itemize}
Some drawbacks -- does not capture nonlinear flow-density relationship of the
fundamental diagram.  \remark{Argue why this is OK for saturated
  urban networks at peak times.  Green waves in SCATS/SCOOT operate on
  principle of a fixed propagation time, so does Gartner's MILP model.
Captures the right level of detail without blowing up size of model.}
}

%%%%%%%%%%%%%%%%%%%%%%%%%%%%%%%%%%%%%%%%%%%%%%%%%%%%%%%%%%%%%%%%%%%%%%%%%%

\comment{
Lin and Wang aim for network-wide control with a MILP but use
a CTM that requires the time-step to be fixed (otherwise changing
time steps would lead to changing cell sizes and require
dynamic traffic realllocation).

Other GA and reinforcement learning methods, but hard to provide
optimality guarantees.

Much work on optimized traffic control focusing on green-wave
signal coordination given predicted demands and are 
non-homogenous, but these are largely focused on arterial
control.

Need a deep lookahead possible with non-homogenous time steps,
but a globally optimized model.  Need to move beyond CTM of
Lin and Wang since this requires fixed time steps, but
still need to model coordination and delays.


Many extensions of CTM for various settings.
...
However, none of these allow non-homogenous modeling.


To this end we introduce the QTM.  Shares aspects of link-based 
and cell-based models while getting delays right (as in existing
MILP control methosd), but allowing for non-homogenous time
steps, unlike existing CTMs or extensions.  Ultimately we
will show that a non-homogenous time step leads to smaller MILPs
and hence improved scalability over a fixed homogenous time
variant.
}

%%%%%%%%%%%%%%%%%%%%%%%%%%%%%%%%%%%%%%%%%%%%%%%%%%%%%%%%%%%%%%%%%%%%%%%%%%

\comment{

daganzo1994cell - original CTM development

daganzo1995cell - original CTM development

alecsandru2011assessment - general CTM survey noting wide use in many areas

xiaojian2010urban - variable CTM, adjustable cell length

jabari2012stochastic - alternative to CTM for modeling random headways

sumalee2011stochastic - stochastic CTM for modeling mean and standard deviation of traffic flow accurately

muralidharan2009freeway - link-node-CTM, better for arteries+freeway

lu2011discrete - lagged CTM, some sort of improvement

kim2002online - situational CTM, defines more cell types to better model boundary conditions of CTM simulation

huang2011traffic - CTM-Urban, claims CTM was for freeways, and defines extensions to better model urban traffic

gomes2006optimal - assymetric CTM (merge imbalance), freeway ramp metering

MISSING: Knoop et al, Network Transmission Model, more macroscopic variation on CTM, reduces under certain conditions


gartner1974optimization, gartner2002arterial, 

el2013multiagent

Control
===

canepa2012exact - not control, but traffic density estimation

gartner1974optimization - green wave coordination MILP?

gartner2002arterial - green wave coordination MILP?  (OPAC?)

han2012link - link-based MILP (road not divided into cells, criticizes CTM as having too many cells, still approximates shockwaves somehow)

el2013multiagent - MARLIN, pairwise Qlearning

lo1998novel - CTM, MILP introduction

lo1999dynamic - CTM, genetic algorithm solution to MILP

lin2004enhanced - CTM-based MILP

he2011pamscod - optimized solutions to MILP signal coordination (travel-time oriented)

he2014multi - multimodal extensions to MILP signal coordination

he2010heuristic - heuristic solutions to MILP signal coordination

smith2013surtrac - SURTRAC, scheduling for green waves

MISSING other optimized control: Also, PRODYN, RHODES... should ideally include these.
}

%%%%%%%%%%%%%%%%%%%%%%%%%%%%%%%%%%%%%%%%%%%%%%%%%%%%%%%%%%%%%%%%%%%%%%%%%%

\comment{

UNUSED TEXT from shelved grant application draft

Congestion is increasing.

Traffic accounts for X hours of people's time and an improvement
of Y\% would lead to Z amount of dollars saved.

Unfortunately, the majority of urban traffic signal coordination is handled by 
1970's and 1980's era systems that neither make effective use of all
data they collect nor rely on modern optimization techniques to attempt
to find an optimal control policy w.r.t.\ some criteria.

(1) while they collect real-time data, they do not make
use of it in a learning manner to build models of traffic flow, which
can later be optimized, and (2) their techniques for
multi-intersection traffic signal coordination are heuristic and
manually tuned.  In this proposal, we leverage the intersection of two
key technologies that will enable a new generation of traffic
controllers: (a) we now have the algorithms and architectures to
crunch large real-time traffic data for online learning of traffic
models and (b) we now have optimization tools like CPLEX and Gurobi
for mixed integer linear programs (MILPs) that can efficiently solve
large-scale optimization problems that would have been intractable
only a decade ago.  Using these technologies, we aim to use the
real-time (Big) data collected from modern traffic control systems to
build a joint hybrid (switching nonlinear) automata model of traffic
signal control and traffic flow and then to used mixed integer linear
programming techniques to iteratively solve for a provably optimal
control strategy for this automaton.

% Continuous time switching automata contributions... bilinear but
% solve with a splitting approach.

%Questions can be (a) improved traffic light control based on
%real-time predictive models of traffic flow, (b) detection of
%anomalies that should be investigated -- lower than normal flow
%%indicating a potential accident / blocked lane, (c) traffic network
%design for improved flow, (d) incentivizing traffic users by selecting
%transit and parking fares to obtain behavior that improves traffic
%flow.

\blankline

{\bf Intellectual Merit:} %\remark{(verbose for now, compress later)}

\begin{itemize}
\item {\it Modeling traffic control as a (nonlinear switching) hybrid automata}
\item {\it Data-driven hybrid automata learning} 
\item {\it Iterative solving of nonlinear switching hybrid automata with mixed integer linear programming}
\end{itemize}

\blankline

{\bf Broader Impact:}

\begin{itemize}
\item {\it Community:} Publications, but also domains in future IPPCs
  to drive future research and comparison of techniques.
\item {\it Engagement/Impact:} Does Australian engagement count,
  e.g. VicRoads?).
\item {\it Education:} MS and PhD training, competitions, courses,
  workshops and tutorials on traffic and data-driven hybrid automata
  model learning.
\end{itemize}

\blankline

{\bf Key Words:} traffic signal control, hybrid automata, smart cities,
Big Data, machine learning

%%%%%%%%%%%%%%%%%%%%%%%%%%%%%%%%%%%%%%%%%%%%%%%%%%%%%%%%%%%%%%%%%%%%%%%
%% Content from a different proposal on smart cities (power, HVAC, etc)
%%%%%%%%%%%%%%%%%%%%%%%%%%%%%%%%%%%%%%%%%%%%%%%%%%%%%%%%%%%%%%%%%%%%%%%

%We are at the cusp of a new era creating smarter, cleaner, more
%efficient cities that leverage real-time data and online optimization
%to intelligently control everything from traffic signals to heating
%and air conditioning in office buildings to the scheduling of
%industrial and consumer demand to coincide with transient peaks in
%production from green energy sources (solar, wind, etc.).

%The key to achieving these tasks lies in the ability to transform
%complex models containing high degrees of concurrency, hybrid (mixed
%discrete and continuous) state, exogenous events

%learned from large quantities of data into actionable decisions in
%real-time --- a task which requires online optimization at a scale
%and speed unprecented in existing online planning work.

%the bottleneck in using it in real-time stems more from the lack of
%scalable optimization.

%Need to be online.

%Problems are large (concurrent, hybrid) structured, require
%exploitation of that structure beyond naive sampling approaches.

%Existing city infrastructure (coordinating traffic lights, optimal
%heating and air conditioning in buildings, ) often operate in a
%manually controlled or otherwise heuristic settings that do not make
%complete use of real-time data that can be made ava

%Data is now available, optimization techniques can now scale to
%thousand and even millions of variables (with decompositions).

}
