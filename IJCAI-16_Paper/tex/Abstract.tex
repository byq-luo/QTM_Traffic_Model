As urban traffic congestion is on the increase worldwide, many cities
are increasingly looking to inexpensive public transit options such as
light rail that operate at street-level and require coordination with
conventional traffic networks and signal control. A major concern in
light rail installation is whether enough commuters will switch to it
to offset the additional constraints it places on traffic signal
control and the resulting decrease in conventional vehicle traffic
capacity. In this paper, we study this problem and ways to mitigate it
through a novel model of optimized traffic signal control subject to
light rail schedule constraints solved in a Mixed Integer Linear
Programming (MILP) framework. Our key results show that while this
MILP approach provides a novel way to optimized fixed-time control
schedules subject to light rail constraints, it also enables a
%there is a substantial impact of light rail on conventional vehicle traffic
%delay using popular fixed-time signal control, our
novel optimized adaptive signal control method that virtually
nullifies the impact of the light rail presence on conventional
traffic flow.  Ultimately this leads to a win-win situation where both
conventional vehicle traffic and light rail commuters benefit through
the application of MILP-based optimization to jointly manage public
transit and conventional traffic networks.
