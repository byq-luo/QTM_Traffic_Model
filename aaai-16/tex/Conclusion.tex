\section{Conclusion}

In this paper, we studied how to mitigate the impact of light rail on
conventional traffic networks via a novel method of optimized traffic
signal control based on Mixed Integer Linear Programming (MILP). Our
key results show that while there is a substantial impact of light
rail on conventional vehicle traffic delay using popular fixed-time
signal control, our novel optimized signal control virtually nullifies
this impact. Ultimately this leads to a win-win situation where both
conventional vehicle traffic and light rail commuters benefit through
the application of MILP-based optimization.

\Omit{
In this paper, we showed how to formulate a novel queue transmission model (QTM)
model of traffic flow with non-homogeneous time steps as a linear program.  We
then proceeded to allow the traffic signals to become discrete variables subject
to a delay minimizing optimization objective and standard traffic signal
constraints leading to a final MILP formulation of traffic signal control with
non-homogeneous time steps.  We
experimented with this novel QTM-based MILP control in a range of traffic networks
and demonstrated that the non-homogeneous MILP formulation
achieved (i) substantially lower delay solutions, (ii) improved per-car delay distributions,
and (iii) more optimal travel times over a longer horizon 
in comparison to the homogeneous MILP formulation with the same number of binary
and continuous variables.
%% NOTE: what bothers me here is ``larger networks'' since we don't directly
%%       compare scalability of the approaches as a function of network size
%%       (i.e., on the x-axis of some graph).  -Scott
%and demonstrated that by exploiting the non-homogeneous time steps supported
%by the QTM, we are able to scale the model up to larger networks whilst maintaining the
%same quality of a homogeneous solution using more binary
%variables.
Altogether, this work represents a
major step forward in the scalability of MILP-based jointly optimized traffic
signal control via the use of a non-homogeneous time traffic models and thus helps
pave the way for fully optimized joint urban traffic signal controllers as an
improved successor technology to existing signal control methods.
}

%We have demonstrated that by exploiting the non-homogeneous time steps
%supported by the QTM, we are able to scale the model up to larger
%networks and using the same number of binary variables as a
%homogeneous time step, and with the same quality of a homogeneous
%solution using more binary variables.
