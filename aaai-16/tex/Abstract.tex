As urban traffic congestion is on the increase worldwide, many cities are
increasingly looking to inexpensive public transit options such as light rail
that operate at street-level and require coordination with conventional traffic
networks and signal control. A major concern in light rail installation is
whether enough commuters will switch to it to offset the additional constraints
it places on traffic signal control and the resulting decrease in conventional
vehicle traffic capacity. In this paper, we study this problem and ways to
mitigate it through the use of a novel method of optimized traffic signal
control based on Mixed Integer Linear Programming (MILP). Our key results show
that while there is a substantial impact of light rail on conventional vehicle
traffic delay using popular fixed-time signal control, our novel optimized
signal control virtually nullifies this impact. Ultimately this leads to a
win-win situation where both conventional vehicle traffic and light rail
commuters benefit through the application of MILP-based optimization to jointly
manage public transit and conventional traffic networks.
