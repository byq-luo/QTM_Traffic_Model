\section{Conclusion}

In this paper, we show how our optimized adaptive traffic signal
control method based on Mixed Integer Linear Programming (MILP) can be
used for mitigating the impact of installing light rail on
conventional traffic networks.  Our experiments with the QTM model of
traffic flow show that our method is able to minimize the impact on
the average delay with respect to fixed-time signal control and also
finds better quality solutions, i.e., solutions with substantially
lower third quartile and maximum observed delay.  These improvements
were also verified in a non-QTM based microsimulation model to
validate performance improvements w.r.t.\ a finer-grained nonlinear
model of traffic flow (e.g., with acceleration and queue-length
effects).
%
%key results show that while there is a substantial impact of light rail on
%conventional vehicle traffic delay using popular fixed-time signal control, our
%novel optimized \authorHighlight{adaptive} signal control virtually nullifies
%this impact.
%
Ultimately our results demonstrate a way to optimize conventional
fixed time controllers w.r.t.\ light rail schedule constraints;
furthermore, when these methods are extended to our fully optimized
adaptive controller, it leads to a win-win situation where both
conventional vehicle traffic and light rail commuters benefit through
the application of MILP-based optimization \emph{regardless} of the
reduction in conventional traffic that opts to take light rail.

For future work, a key question to resolve is how large we can scale
the traffic and light rail network before we need to investigate
decomposition-based approaches to scaling the solution 
(e.g, 
MILP-based methods like dual decomposition or region-based
traffic network partitioning schemes).
%used in a variety of conventional
%traffic control methods.
Future
work should also examine the (online) learnability of the QTM model
from different traffic sensor data ranging from conventional inductive
(double) loop counters through to video feeds.  Finally, noting that
the nonlinear microsimulation model offers a higher-fidelity model of actual
traffic behavior, future work should consider expanding the QTM to
model nonlinear traffic
flows~\cite{lu2011discrete,muralidharan2009freeway,kim2002online,huang2011traffic}
and investigating the benefits of nonlinear optimization in this
setting relative to the existing QTM.

Notwithstanding this important future work, our key results
demonstrate for the first time the potential of MILP-based traffic
signal control approaches to
%optimize fixed-time control schedules for
%existing legacy traffic controllers subject to light rail constraints.
%Our results further enable a novel future generation of optimized
%adaptive signal controllers that virtually
nullify the impact of installing light rail on
conventional traffic.  Consequently, the use
of such optimized controllers as proposed here could
%substantially
%mitigate concerns of the impact of light rail on conventional
%traffic networks and
remove this critical public concern of light rail installation and thus 
positively impact the
environment, urban productivity, and commute time reductions for both
conventional traffic and public transit riders as a result.

\Omit{
In this paper, we showed how to formulate a novel queue transmission model (QTM)
model of traffic flow with non-homogeneous time steps as a linear program.  We
then proceeded to allow the traffic signals to become discrete variables subject
to a delay minimizing optimization objective and standard traffic signal
constraints leading to a final MILP formulation of traffic signal control with
non-homogeneous time steps.  We
experimented with this novel QTM-based MILP control in a range of traffic networks
and demonstrated that the non-homogeneous MILP formulation
achieved (i) substantially lower delay solutions, (ii) improved per-car delay distributions,
and (iii) more optimal travel times over a longer horizon 
in comparison to the homogeneous MILP formulation with the same number of binary
and continuous variables.
%% NOTE: what bothers me here is ``larger networks'' since we don't directly
%%       compare scalability of the approaches as a function of network size
%%       (i.e., on the x-axis of some graph).  -Scott
%and demonstrated that by exploiting the non-homogeneous time steps supported
%by the QTM, we are able to scale the model up to larger networks whilst maintaining the
%same quality of a homogeneous solution using more binary
%variables.
Altogether, this work represents a
major step forward in the scalability of MILP-based jointly optimized traffic
signal control via the use of a non-homogeneous time traffic models and thus helps
pave the way for fully optimized joint urban traffic signal controllers as an
improved successor technology to existing signal control methods.
}

%We have demonstrated that by exploiting the non-homogeneous time steps
%supported by the QTM, we are able to scale the model up to larger
%networks and using the same number of binary variables as a
%homogeneous time step, and with the same quality of a homogeneous
%solution using more binary variables.
